\begin{table}[htbp]\centering \def\sym#1{\ifmmode^{#1}\else\(^{#1}\)\fi} \caption{Treatment effects across domains -- Risk aversion} \label{tab:REG-risk} \maxsizebox*{\textwidth}{\textheight}{ \begin{threeparttable} \begin{tabular}{l*{6}{c}} \toprule
          &\multicolumn{3}{c}{Within experiments}&\multicolumn{3}{c}{Across experiments}\\\cmidrule(lr){2-4}\cmidrule(lr){5-7}
          &\multicolumn{1}{c}{(1)}&\multicolumn{1}{c}{(2)}&\multicolumn{1}{c}{(3)}&\multicolumn{1}{c}{(4)}&\multicolumn{1}{c}{(5)}&\multicolumn{1}{c}{(6)}\\
          &\multicolumn{1}{c}{TSST-G}&\multicolumn{1}{c}{CPT}&\multicolumn{1}{c}{CENT}&\multicolumn{1}{c}{\specialcell{CPT vs.\\TSST-G}}&\multicolumn{1}{c}{\specialcell{CENT vs.\\TSST-G}}&\multicolumn{1}{c}{\specialcell{CENT\\vs. CPT}}\\
\midrule
Coefficient of relative risk aversion&         &     0.12&-0.22\sym{*}&         &         &    -0.33\\
          &         &   (0.19)&   (0.12)&         &         &   (0.23)\\
          &         &         &         &         &         &         \\
\bottomrule \end{tabular} \begin{tablenotes}[flushleft] \footnotesize \item \emph{Notes:} Columns 1-3 report treatment effect estimates on each row variable for TSST-G, CPT, and CENT, respectively. Columns 4-6 report differences in the treatment effect across experiments. Standard errors are in parentheses and FDR-adjusted minimum \(q\)-values are in brackets. Stars on the coefficient estimates indicate significance with naive \(p\)-values. The bottom row reports \(p\)-values of a joint test across row variables. * denotes significance at 10 pct., ** at 5 pct., and *** at 1 pct. \end{tablenotes} \end{threeparttable} } \end{table}
