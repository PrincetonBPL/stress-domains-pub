\begin{table}[htbp]\centering \def\sym#1{\ifmmode^{#1}\else\(^{#1}\)\fi} \caption{NLS estimates of quasi-hyperbolic discounting for CENT} \label{tab:REG-nls} \maxsizebox*{\textwidth}{\textheight}{ \begin{threeparttable} \begin{tabular}{l*{2}{c}} \toprule
          &\multicolumn{1}{c}{Treatment}&\multicolumn{1}{c}{Control}\\
\midrule
\(\beta\) &    0.604&    0.658\\
          &  (0.017)&  (0.018)\\
\(\delta\)&    0.261&    0.306\\
          &  (0.016)&  (0.031)\\
\midrule \(\mathrm{H}_0: \beta = 1\)&    0.000&    0.000\\
\(\mathrm{H}_0: \delta = 1\)&    0.000&    0.000\\
\(\mathrm{H}_0: \beta_T = \beta_C\)&    0.040&         \\
\(\mathrm{H}_0: \delta_T = \delta_C\)&    0.209&         \\
\bottomrule \end{tabular} \begin{tablenotes}[flushleft] \footnotesize \item \emph{Notes:} This table presents estimates for the \(\beta-\delta\) model for the treatment and control groups of the CENT experiment. Standard errors are clustered at the session level. We report \(p\)-values for each hypothesis test. \end{tablenotes} \end{threeparttable} } \end{table}
